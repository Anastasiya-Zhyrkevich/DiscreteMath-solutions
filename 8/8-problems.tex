\documentclass{article}
\usepackage[margin=0.25in]{geometry}
\usepackage[utf8]{inputenc}


\usepackage[T2A]{fontenc}
\usepackage[utf8]{inputenc}
\usepackage[russian]{babel}
\usepackage{amssymb}
\usepackage{multicol}
\usepackage{amsmath}

\begin{document}

\textbf{4.}

\textbf{(в)}

Разберем первую часть 

$$\exists x (P(x) \wedge (P(x) \thicksim (Q(x) \vee  \overline{P} (x) ))) = True$$

Значит, существует $x_0 \in M$, такое, что при подстановке получается истинное выражение 

$$P(x_0) \wedge (P(x_0) \thicksim (Q(x_0) \vee  \overline{P} (x_0) )) = True$$

Решим это как уравнение. Составим систему:

$$\begin{cases} P(x_0) = True \\ 
P(x_0) \thicksim (Q(x_0) \vee  \overline{P} (x_0) ) = True  \end{cases} $$

Подставляем первое во второе

$$\begin{cases} P(x_0) = True \\ Q(x_0) \vee  False = True \end{cases} $$

$$\begin{cases} P(x_0) = True \\ Q(x_0) = True \end{cases} $$

Из полученного следует, что существует $x_0$ такое, что

$$P(x_0) \wedge Q(x_0) = True$$

Значит,

$$\exists x (P(x) \wedge Q(x)) = True$$

Что и требовалось доказать


\textbf{(г)}

Рассмотрим первую часть 

$$ \forall x (\overline{P} (x) \rightarrow (P(x) \vee \overline{ \overline{Q} (x) \rightarrow P (x) })) = False $$

Преобразуем высказывание

$$ \forall x (P(x) \vee (P(x) \vee \overline{ Q (x) \vee P (x) })) = False $$

$$ \forall x (P(x) \vee ( \overline{ Q } (x) \wedge \overline{P} (x) )) = False $$

Получаем, что существует $x_0 \in M$ такое, что 

$$\begin{cases} P(x_0) = False \\ \overline{ Q } (x_0) \wedge \overline{P} (x_0) = False \end{cases} $$

$$\begin{cases} P(x_0) = False \\  Q  (x_0) = True \end{cases} $$

Значит,

$$ \forall x P(x) = False б \exists x Q(x) = True$$

Что и требовалось доказать


\textbf{5}

\textbf{(a)}

Рассмотрим внимательно последнюю часть конньюнкции

$$ \forall x \exists y P(x, y)$$ 

В силу того, что $M$ - конечно, получаем, что $$ \forall x \exists y P(x, y) = False$$

Т.к. если зафиксировать $x_0 = n$, то для этого значения не существует значения $ y$, $x_0 < y$.

Значит, и все высказывание, данное в условии - ложное.


\textbf{(б)}

Рассуждения аналогичны пункту 5а. Однако в данном примере множество значений - бесконечное. А значит,

$$ \forall x \exists y P(x, y) = True$$ 

Остальные части конъюнкции доказываются просто. 

Значит, и все высказывание, данное в условии - истинное.

\textbf{(в)}

$$P(y) = "2 делит 3" = False$$

Если подставить $x_0 = 2$, то получим, что $P(x_0) = "2 делит 3" = True$

А значит, $\exists x P(x) = True$. 

Собирая результаты обоих частей импликации получаем, что данное высказывание - ложное. 

\textbf{(г)}

Обе части равносильности - ложные, т.к. 

$$\forall P(x) = \forall x "3 делит x" = False$$

$$\forall Q(x) = \forall x "2 делит x" = False$$

А значит, получаем, что данное исходное высказывание - истинное.


\textbf{8} 

\textbf{(a)}

1. Пусть левая части равносильности - истинна. Тогда для любого значения $x = x_0 \in M$ верно,

$$P(x_0) \wedge Q(x_0) = True$$

Из этого следует, что $P, Q$ - тождественные истины на множестве $M$. Получаем, что 

$$\forall x P(x) = True, \forall x Q(x) = True$$

Левая часть высказывания получается также истинной. 

$$(\forall x P(x)) \wedge (\forall x Q(x)) = True$$

2. Пусть левая части равносильности - ложная.

Тогда существует $x_0$ такое, что 

$$P(x_0) \wedge Q(x_0) = False$$

Значит, 
$$P(x_0) = False, Q(x_0) = False$$

Тогда

$$\forall x P(x) = False, \forall x Q(x) = False$$ 

Что и требовалось.

\textbf{(б)}

Рассмотри левую часть равносильности
$$\overline{ \exists x P ( x ) }  $$

1. Пусть она истинна. Тогда

$$\exists x P(x) = False$$ 

Это значит, что ни при каких значениях $x$ выражение $P(x)$ не обращается в истинное. 

Другими словами, для любых значений - $P(x) = False$. Если записать математически, то получим, что 

$$ \forall x \overline{ P (x) } $$

2. Пусть левая часть равносильности ложная. Тогда 

$$\exists x P(x) = True$$

Другими словами, $$P(x_0) = True$$. Или $$\overline{P} (x_0) = False$$

Тогда выражение 
$$\forall x \overline{P} (x) $$ является ложным, потому что есть значение, которое обращается все высказывание в ложное. 



\end{document}
