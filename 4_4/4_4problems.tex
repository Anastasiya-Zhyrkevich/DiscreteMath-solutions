\documentclass{article}
\usepackage[margin=0.25in]{geometry}
\usepackage[utf8]{inputenc}


\usepackage[T2A]{fontenc}
\usepackage[utf8]{inputenc}
\usepackage[russian]{babel}
\usepackage{amssymb}
\usepackage{multicol}
\usepackage{amsmath}
\usepackage{tikz}

\begin{document}

\textbf{1}
Задача на Дирихле

Каждый день на протяжении четырехнедельного отпуска отдыхающий играл по крайней мере одну партию в шахматы. Общее число сыгранных партий не превышает $40$. Докажите, что надйтеся промежуток времени, состоящий из последовательный дней, в течении которых было сыграно ровно  $15 $ партий. 

$$A = [1, 4, [2, 5, 7, 1,] 3, 4  ... ]$$

Есть массив из 28 элементов. Нужно найти подмассив сумма элементов = 15.

Префикс сумм -- массив размера 29.

$$S_0 = 0$$
$$ S_{i + 1} = sum_{j = 0}^{j = i} A_j$$ 

$$ S = [0, 1, 5, 7, 12, 19, 20, .. ] $$ 

$$ S_0 = 0$$
$$S_1 = A_0 = 1$$
$$S_2 = A_0 + A_1 = 5$$
$$S_3 = A_0 + A_1 + A_2 = 7$$
$$S_4 = A_0 + A_1 + A_2 + A_3 = 12$$
$$S_5 = A_0 + A_1 + A_2 + A_3 + A_4 = 19$$
$$S_6 = A_0 + A_1 + A_2 + A_3 + A_4 + A_5= 20$$


$$ A_2 + A_3 + A_4  + A_5 = (A_0 + A_1 + A_2 + A_3 + A_4 + A_5) - (A_0 + A_1 ) = S_6 - S_2$$

Сумма на любом подотрезке вычисляется быстро !!


$$ A_i + A_{i + 1} + ... + A_{j} = S_{j + 1} - S_i$$

$$ S = [0, 1, 5, 7, 12, 19, 20, .. ] $$ 
Строите массив $S^`$ ОСТАТКИ ОТ ДЕЛЕНИЯ НА 15

$$ S = [0, 1, 5, 7, 12, 4, 5, .. ] $$ 

Получили две пятерки

$$ A_0 + A_1 $$ - делилась на 15 с остатком 5.

$$A_0 + A_1 + A_2 + A_3 + A_4 + A_5$$ - делилась на 15 с остатком 5.

Если рассмотреть разнность этих велчин, то получится что-то последовательное, что делится на 15.

\newpage

Вопрос на понимание

Сложность вычисление 

Задача: Вам дан массив. Нужно выписать в ряд все суммы на подотрезках. 

$$ A = [1, 0, 7]$$

$$[1, 1, 8, 0, 7, 7]$$

$$[1]$$
$$[1 + 0]$$
$$[1 + 0 + 7]$$

$$[0]$$
$$[0 + 7]$$

$$[7]$$

За сколько Вы умеете строить ? - $n ^2$

$$A = [1,2, 3, 4, 5,6, 7, 8,9]$$

Многократное обращение.

$n$ - размер $A$
$m$ - запросов.
 
Решение :
Подсчитать один раз S

$func(i, j) = A_i + A_{i+ 1} + ... A_j = S_{j + 1} - S_i$

$n + m$ операций


А если бы честно считалли сумму каждый раз, то было бы 
$ m \cdot n$ операций

\newpage

\textbf{2} 
Функции - таблица 
Сюрьективня, иньективная, биективная

$X, Y = R (R \rightarrow R)$

\begin{center}
\begin{tabular}{ c c c c }
  Функция & Инъективна & Сюрьективна & Биективная \\
  $f(x) = sin(x)$ & - &  - &  - \\
  $g(x) = 2\cdot x + 5$ & + & + & + \\
  $x +1$ & +  & +  & + \\
  $x^2 - 3x +  10$ & - & -  &  - \\
  $x^3 + 5x^2 - 8$ & -  & + & - \\
  $x^3$ &  +  &  +  & +  \\
  $x^4 - 3x^3 + 1$ & - &  - &  - \\
  $|x|$ & -  & - & - \\
  $\frac{x + 1}{x - 1}$ & +  & - & - \\
\end{tabular}
\end{center}

$(R \setminus \{1\} \rightarrow R)$





Чтобы подставить "$-$" иньективность
$sin(0) = sin(2\pi) = 0$

$- b / 2 = 3/2 $
$a = 1, b = 2: 1 - 3 + 10 = 4 - 6 + 10$

\bigskip

Чтобы подставить "$-$" сюръевтина

$sin(?) = 2$

$x ^2 - 3x + 10 = -100 (x = ??)$ 


$\frac{x + 1}{x - 1} = 1  (x = ??)$


\bigskip

$2a + 5 = 2b + 5 \rightarrow a = b $

$\frac{a + 1}{a - 1} = \frac{b + 1}{b - 1}$

$(a + 1)(b - 1) = (b + 1)(a - 1)$

$$ab + b - a - 1 = ab + a - b - 1$$
$$2b = 2a$$
\newpage

$R^2 \rightarrow R$
\begin{center}
\begin{tabular}{ c c c c }
  Функция & Инъективна & Сюрьективна & Биективная \\
  $f(x, y) = x + y$ & -  & +  & - \\
  $x^2 + y^2$ & - & - & - \\
\end{tabular}
\end{center}



\newpage

\textbf{3}
Отображение

Дано множество 
$$A = \{10, 11, 12, ... 20\}$$

$$f : A \rightarrow A$$

$$f(A) = \{(10, 12), (12, 14), (15, 11), (20, 17), (13, 17) \}$$

Найти $f^{-1}(\{13, 17\})$

$f^{-1}(\{13, 17\}) = \{13, 20 \}$

$f^{-1}(\{13\}) = \emptyset$

$f^{-1}(\{17\}) = \{13, 20 \}$

\newpage

\textbf{4} 
Задано множество $$A = \{ 1, 2, 7, \{5\}, \{1, \{ 3,4 ,5 \}\}, \{\emptyset \}\}$$

$$A = \{ 1, 2, 7, a, b, c\}$$

Чему равно $|2^A|$

\begin{itemize}
    \item $X = \{a, b, c\}$
    
    $2^X = \{\emptyset, \{a\}, \{b\}, \{c\}, \{a, b\}, \{a, c\}, \{b, c\}, \{a, b, c\}\}$
    $|2^X| = 2 ^ {|X|} = 2 ^ 3$  
    
    
    \item $X = \{\emptyset, \{ \emptyset \}\}$
    
    Замечание: $\emptyset \neq \{ \emptyset \}$
    
    $2^X = \{ \emptyset, \{\emptyset\}, \{\{ \emptyset \}\}, \{\emptyset, \{  \emptyset \}\}\}$
    
\end{itemize}


$|2^A| = 2^6$

\newpage

\textbf{5}

Даны множества

$A = \{ 3, 4, ... 10 \}$

$B = \{ 8, 9, ... 15 \}$

$C = \{ 15, 16, 20 \}$

$D = \{ 16, 17, 18, 19, 20, 21\}$

Найти 

\begin{center}
\begin{tabular}{ c c c c }
  Функция & Значение & Примеры \\
  $|A \setminus B|$ & 5  &  \{3, 4, 5, 6, 7 \}\\
  $| A \oplus B|$ & 10 &  \{3, 4, 5, 6, 7, 11, 12, 13, 14, 15 \}\\
  $| A \triangle B|$ & 10 & \\
  $| A \times B|$ & $8 \cdot 8 = 64$ &  $\{(3, 9)\}$ \\
  $| (A \cup B) \times (C \cap D)|$ & $13 \cdot 2 = 26$ & \\
\end{tabular}
\end{center}


\newpage

\textbf{6} 

Множества. Докажите следующие тождества, используя равносильные преобразования. (номер 14 из Сборника задач Дугинова О.И.)

\begin{itemize}
    \item $A \cap B = A \setminus (A \setminus B)$
    \item $A \cup (\overline{A} \cap B) = A \cup B$
    \item $(A \cup B) \cap (A \cup \overline{B}) = A$
\end{itemize}

\textbf{Решение 1}

\begin{equation} 
\begin{split}
A \setminus (A \setminus B) 
& = A \cap (\overline{A \setminus B}) \\
& = A \cap (\overline{A \cap \overline{B} }) \\
& = A \cap (\overline{A} \cup B ) \\
& = (A \cap \overline{A}) \cup (A \cap B ) \\
& = \emptyset \cup (A \cap B ) \\
& = A \cap B  \\
\end{split}
\end{equation}


\textbf{Решение 2}

\begin{equation} 
\begin{split}
A \cup (\overline{A} \cap B) 
& = (A \cup \overline{A}) \cap (A \cup B) \\
& = U \cap (A \cup B) \\
& = A \cup B \\
\end{split}
\end{equation}

\textbf{Решение 3}

\begin{equation} 
\begin{split}
(A \cup B) \cap (A \cup \overline{B}) 
& = ((A \cup B) \cap A) \cup ((A \cup B) \cap \overline{B}) \\
& = (A \cap A) \cup (B \cap A) \cup (A \cap \overline{B}) \cup (B \cap \overline{B}) \\
& = A \cup (A \cap \overline{B}) \cup \emptyset \\
& = A  \\
\end{split}
\end{equation}

\end{document}
