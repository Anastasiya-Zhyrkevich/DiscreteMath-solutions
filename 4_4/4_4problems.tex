\documentclass{article}
\usepackage[margin=0.25in]{geometry}
\usepackage[utf8]{inputenc}


\usepackage[T2A]{fontenc}
\usepackage[utf8]{inputenc}
\usepackage[russian]{babel}
\usepackage{amssymb}
\usepackage{multicol}
\usepackage{amsmath}
\usepackage{tikz}

\begin{document}

\textbf{1}
Задача на Дирихле

\textbf{2} 
Функции - таблица 
Сюрьективня, иньективная, биективная

$X, Y = R (R \rightarrow R)$

\begin{center}
\begin{tabular}{ c c c c }
  Функция & Инъективна & Сюрьективна & Биективная \\
  $f(x) = sin(x)$ &  &  & \\
  $g(x) = 2\cdot x + 5$ &  &  & \\
  $x +1$ &  &  & \\
  $x^2 - 3x +  10$ &  &  & \\
  $x^3 + 5x^2 - 8$ &  &  & \\
  $x^3$ &  &  & \\
  $x^4 - 3x^3 + 1$ &  &  & \\
  $|x|$ &  &  & \\
  $\frac{x + 1}{x - 1}$ &  &  & \\
\end{tabular}
\end{center}

$R^2 \rightarrow R$
\begin{center}
\begin{tabular}{ c c c c }
  Функция & Инъективна & Сюрьективна & Биективная \\
  $f(x, y) = x + y$ &  &  & \\
  $x^2 + y^2$ &  &  & \\
\end{tabular}
\end{center}

\textbf{3}
Отображение

Дано множество 
$$A = \{10, 11, 12, ... 20\}$$

$$f : A \rightarrow A$$

$$f(A) = \{(10, 12), (12, 14), (15, 11), (20, 17), (13, 17) \}$$

Найти $f^{-1}(\{13, 17\})$

\textbf{4} 
Задано множество $$A = \{ 1, 2, 7, \{5\}, \{1, \{ 3,4 ,5 \}\}, \{\emptyset \}\}$$

Чему равно $|2^A|$

\textbf{5}

Даны множества

$A = \{ 3, 4, ... 10 \}$
$B = \{ 8, 9, ... 15 \}$
$C = \{ 15, 16, 20 \}$
$D = \{ 16, 17, 18, 19, 20, 21\}$

Найти 

\begin{center}
\begin{tabular}{ c c c c }
  Функция & Значение & Примеры \\
  $|A \setminus B|$ &  & \\
  $| A \oplus B|$ &  & \\
  $| A \triangle B|$ &  & \\
  $| A \times B|$ &  & \\
  $| (A \cup B) \times (C \cap D)|$ &  & \\
\end{tabular}
\end{center}


\textbf{6} 

Множества. Докажите следующие тождества, используя равносильные преобразования. (номер 14 из Сборника задач Дугинова О.И.)

\begin{itemize}
    \item $A \cap B = A \setminus (A \setminus B)$
    \item $A \cup (\overline{A} \cap B) = A \cup B$
    \item $(A \cup B) \cap (A \cup \overline{B}) = A$
\end{itemize}

\textbf{Решение 1}

\begin{equation} 
\begin{split}
A \setminus (A \setminus B) 
& = A \cap (\overline{A \setminus B}) \\
& = A \cap (\overline{A \cap \overline{B} }) \\
& = A \cap (\overline{A} \cup B ) \\
& = (A \cap \overline{A}) \cup (A \cap B ) \\
& = \emptyset \cup (A \cap B ) \\
& = A \cap B  \\
\end{split}
\end{equation}


\textbf{Решение 2}

\begin{equation} 
\begin{split}
A \cup (\overline{A} \cap B) 
& = (A \cup \overline{A}) \cap (A \cup B) \\
& = U \cap (A \cup B) \\
& = A \cup B \\
\end{split}
\end{equation}

\textbf{Решение 3}

\begin{equation} 
\begin{split}
(A \cup B) \cap (A \cup \overline{B}) 
& = ((A \cup B) \cap A) \cup ((A \cup B) \cap \overline{B}) \\
& = (A \cap A) \cup (B \cap A) \cup (A \cap \overline{B}) \cup (B \cap \overline{B}) \\
& = A \cup (A \cap \overline{B}) \cup \emptyset \\
& = A  \\
\end{split}
\end{equation}

\end{document}
