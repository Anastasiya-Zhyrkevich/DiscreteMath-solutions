%% This Beamer template is based on the one found here: https://github.com/sanhacheong/stanford-beamer-presentation, and edited to be used for Stanford ARM Lab

\documentclass[10pt]{beamer}
%\mode<presentation>{}

\usepackage{media9}
\usepackage{amssymb,amsmath,amsthm,enumerate}
\usepackage[utf8]{inputenc}
\usepackage{array}
\usepackage[parfill]{parskip}
\usepackage{graphicx}
\usepackage{caption}
\usepackage{subcaption}
\usepackage{bm}
\usepackage{amsfonts,amscd}
\usepackage[]{units}
\usepackage{listings}
\usepackage{multicol}
\usepackage{multirow}
\usepackage{tcolorbox}
\usepackage{physics}
\usepackage[T2A]{fontenc}
\usepackage[utf8]{inputenc}
\usepackage[russian]{babel}

% Enable colored hyperlinks
\hypersetup{colorlinks=true}

% The following three lines are for crossmarks & checkmarks
\usepackage{pifont}% http://ctan.org/pkg/pifont
\newcommand{\cmark}{\ding{51}}%
\newcommand{\xmark}{\ding{55}}%

% Numbered captions of tables, pictures, etc.
\setbeamertemplate{caption}[numbered]

%\usepackage[superscript,biblabel]{cite}
\usepackage{algorithm2e}
\renewcommand{\thealgocf}{}

% Bibliography settings
\usepackage[style=ieee]{biblatex}
\setbeamertemplate{bibliography item}{\insertbiblabel}
\addbibresource{references.bib}

% Glossary entries
\usepackage[acronym]{glossaries}
\newacronym{ML}{ML}{machine learning}
\newacronym{HRI}{HRI}{human-robot interactions}
\newacronym{RNN}{RNN}{Recurrent Neural Network}
\newacronym{LSTM}{LSTM}{Long Short-Term Memory}


\theoremstyle{remark}
\newtheorem*{remark}{Remark}
\theoremstyle{definition}

\newcommand{\empy}[1]{{\color{darkorange}\emph{#1}}}
\newcommand{\empr}[1]{{\color{cardinalred}\emph{#1}}}
\newcommand{\examplebox}[2]{
\begin{tcolorbox}[colframe=darkcardinal,colback=boxgray,title=#1]
#2
\end{tcolorbox}}

\usetheme{Stanford} 
\input{./style_files_stanford/my_beamer_defs.sty}


% commands to relax beamer and subfig conflicts
% see here: https://tex.stackexchange.com/questions/426088/texlive-pretest-2018-beamer-and-subfig-collide
\makeatletter
\let\@@magyar@captionfix\relax
\makeatother

\title[Дискретная математика]{Логические операции над высказываниями}
%\subtitle{Subtitle Of Presentation}

%\beamertemplatenavigationsymbolsempty

\begin{document}


\author[BSU]{
	\begin{tabular}{c} 
	\Large
	\\
    \footnotesize \href{mailto:anastasiya.zhyrkevich@yandex.ru}{anastasiya.zhyrkevich@yandex.ru}
\end{tabular}
\vspace{-4ex}}

\institute{
	\vskip 5pt
	\vskip 5pt
	Дискретная математика\\
	Семинар 2\\
	\vskip 3pt
}

\date{15 February, 2020}
% \date{\today}

\begin{noheadline}
\begin{frame}\maketitle\end{frame}
\end{noheadline}

\setbeamertemplate{itemize items}[default]
\setbeamertemplate{itemize subitem}[circle]

\begin{frame}
	\frametitle{Краткое содержание} % Table of contents slide, comment this block out to remove it
	\tableofcontents % Throughout your presentation, if you choose to use \section{} and \subsection{} commands, these will automatically be printed on this slide as an overview of your presentation
\end{frame}

\section{Проверка домашнего задания}
\begin{frame}[allowframebreaks]
\frametitle{Домашнего задание}

TBD

\end{frame}


\section{Решение уравнений}
\begin{frame}[allowframebreaks]
\frametitle{Задание 4 (а)}

Задание: 

Решите следующее логическое уравнение: 

$$ (A \to C) \cdot \overline{((B \to C) \to ((A \vee B) \to C))} = True $$

\framebreak 

Задание: 

Решите следующее логическое уравнение: 

$$ (A \to C) \cdot \overline{((B \to C) \to ((A \vee B) \to C))} = True $$

Решение: 

Составим систему

\begin{cases} 
A \to C = T \\ 
\overline{((B \to C) \to ((A \vee B) \to C))} = T 
\end{cases} 

\framebreak 

\begin{cases} 
A \to C = T \\ 
\overline{((B \to C) \to ((A \vee B) \to C))} = T 
\end{cases} 

Первое - неинфомативное 

Второе - работаем

$$\overline{((B \to C) \to ((A \vee B) \to C))} = T $$

\framebreak 

$$(B \to C) \to ((A \vee B) \to C) = F$$

Единственный случай: 

Система : 

\begin{cases} 
B \to C = T \\
(A \vee B) \to C = F
\end{cases} 

\framebreak 

\begin{cases} 
B \to C = T \\
(A \vee B) \to C = F
\end{cases} 

\begin{cases} 
B \to C = T \\
A \vee B = T \\ 
C = F
\end{cases} 

\framebreak 

\begin{cases} 
B \to C = T \\
A \vee B = T \\ 
C = F
\end{cases} 

\begin{cases} 
B = F \\
A \vee B = T \\ 
C = F
\end{cases} 

\begin{cases} 
B = F \\
A = T \\ 
C = F
\end{cases}

\framebreak 

\begin{cases} 
B = F \\
A = T \\ 
C = F
\end{cases}

Возвращаемся к $A \to C = T$

Оно неверное. 

Ответ: Решений нет. 

\framebreak

Резюме решения: 

\begin{cases} 
A \to C = T \\ 
\overline{((B \to C) \to ((A \vee B) \to C))} = T 
\end{cases} 

\begin{cases} 
A \to C = T \\ 
(B \to C) \to ((A \vee B) \to C) = F 
\end{cases} 

\begin{cases} 
A \to C = T \\ 
B \to C = T \\
(A \vee B) \to C = F
\end{cases} 

\begin{cases} 
A \to C = T \\ 
B \to C = T \\
A \vee B = T \\
C = F
\end{cases} 
\end{frame}

\begin{frame}[allowframebreaks]
\frametitle{Задание 4 (б)}

Задание: 

Решите следующее логическое уравнение: 

$$ ((A \cdot B) \sim C) \to (C \vee \overline{A}) = False  $$

\framebreak 

Задание: 

Решите следующее логическое уравнение: 

$$ ((A \cdot B) \sim C) \to (C \vee \overline{A}) = False  $$

Решение: 

\begin{cases} 
((A \cdot B) \sim C) = T \\ 
C \vee \overline{A} = F \\
\end{cases} 

\framebreak

\begin{cases} 
((A \cdot B) \sim C) = T \\ 
C \vee \overline{A} = F \\
\end{cases} 


\begin{cases} 
((A \cdot B) \sim C) = T \\ 
C = F \\ 
\overline{A} = F \\
\end{cases} 

\begin{cases} 
((A \cdot B) \sim C) = T \\ 
C = F \\ 
A = T \\
\end{cases}

\framebreak 

\begin{cases} 
A \cdot B = F\\ 
C = F \\ 
A = T \\
\end{cases}

\begin{cases} 
B = F\\ 
C = F \\ 
A = T \\
\end{cases}

\framebreak 

Ответ: 

\begin{cases} 
B = F\\ 
C = F \\ 
A = T \\
\end{cases}

\end{frame}

\begin{frame}[allowframebreaks]
\frametitle{Задание 4 (г)}

Задание: 

Решите следующее логическое уравнение: 

$$ (\overline{(A \sim B)} \cdot \overline{(A \sim C)}) \to \overline{(A \sim (B \cdot D))}  = False $$

\framebreak 

Задание: 

Решите следующее логическое уравнение: 

$$ (\overline{(A \sim B)} \cdot \overline{(A \sim C)}) \to \overline{(A \sim (B \cdot D))}  = False $$

Решение: 

\begin{cases} 
(\overline{(A \sim B)} \cdot \overline{(A \sim C)}) = T\\ 
\overline{(A \sim (B \cdot D))} = F\\ 
\end{cases}

\framebreak

\begin{cases} 
(\overline{(A \sim B)} \cdot \overline{(A \sim C)}) = T\\ 
\overline{(A \sim (B \cdot D))} = F\\ 
\end{cases}

\begin{cases} 
\overline{(A \sim B)} = T \\ 
\overline{(A \sim C)}  = T \\ 
A \sim (B \cdot D) = T\\ 
\end{cases}

\framebreak

\begin{cases} 
\overline{(A \sim B)} = T \\ 
\overline{(A \sim C)} = T \\ 
A \sim (B \cdot D) = T\\ 
\end{cases}

\begin{cases} 
A \sim B = F \\ 
A \sim C = F \\ 
A \sim (B \cdot D) = T\\ 
\end{cases}

\framebreak

\begin{cases} 
A \sim B = F \\ 
A \sim C = F \\ 
A \sim (B \cdot D) = T\\ 
\end{cases}

Рассмотрим 2 случая 

Пусть $A = T$, Тогда 

\begin{cases} 
A = T \\
B = F \\ 
C = F \\ 
B \cdot D = T\\ 
\end{cases}

Противоречие

\framebreak

\begin{cases} 
A \sim B = F \\ 
A \sim C = F \\ 
A \sim (B \cdot D) = T\\ 
\end{cases}

Рассмотрим второй случай

Пусть $A = F$, Тогда 

\begin{cases} 
A = F \\
B = T \\ 
C = T \\ 
B \cdot D = T\\ 
\end{cases}

\begin{cases} 
A = F \\
B = T \\ 
C = T \\ 
D = F\\ 
\end{cases}


\end{frame}


\section{Полезные преобразования}
\begin{frame}[allowframebreaks]
\frametitle{Преобразования}

Преобразования: 
\begin{enumerate}
    \item $A \to B = \overline{A} \vee B$
    \item $A \cdot (B \vee C) = (A \cdot B) \vee (A \cdot C)$
    
    \item $\overline{A} \vee (A \cdot B) = \overline{A} \vee B$
\end{enumerate}

\end{frame}

\section{Доказательство равносильностей}
\begin{frame}[allowframebreaks]
\frametitle{Разминка}

Доказать $$\overline{A} \vee (A \cdot B) = \overline{A} \vee B$$

\end{frame}


\section{Доказательство равносильностей}
\begin{frame}[allowframebreaks]
\frametitle{Задание 5 (а)}

Задание: 

Докажите следующую равносильность без использования таблиц истинности: 

$$ (A \cdot (B \vee \overline{C})) \vee \overline{A} \vee (B \cdot C) \vee (A \cdot \overline{C}) \equiv \overline{A} \vee B \vee \overline{C} $$

\framebreak

Задание: 

Докажите следующую равносильность без использования таблиц истинности: 

$$ (A \cdot (B \vee \overline{C})) \vee \overline{A} \vee (B \cdot C) \vee (A \cdot \overline{C}) \equiv \overline{A} \vee B \vee \overline{C} $$

Решение: 

Преобразуем левую часть

$$(A \cdot (B \vee \overline{C})) \vee \overline{A} \vee (B \cdot C) \vee (A \cdot \overline{C}) $$

\framebreak

$$(A \cdot (B \vee \overline{C})) \vee \overline{A} \vee (B \cdot C) \vee (A \cdot \overline{C}) \equiv $$

$$[ \boldsymbol{FORMULA:}  A \cdot (B \vee C) = (A \cdot B) \vee (A \cdot C) ] $$


$$(A \cdot B) \vee (A \cdot \overline{C}) \vee \overline{A} \vee (B \cdot C) \vee (A \cdot \overline{C}) \equiv $$

\framebreak

$$(A \cdot B) \vee (A \cdot \overline{C}) \vee \overline{A} \vee (B \cdot C) \vee (A \cdot \overline{C}) \equiv $$

(Одинаковые конъюнкции)

$$(A \cdot B) \vee (A \cdot \overline{C}) \vee \overline{A} \vee (B \cdot C) \equiv $$

\framebreak

$$(A \cdot B) \vee \boldsymbol{(A \cdot \overline{C}) \vee \overline{A}} \vee (B \cdot C) \equiv $$

$$[ \boldsymbol{FORMULA:}  \overline{A} \vee (A \cdot B) = \overline{A} \vee B ]$$

$$(A \cdot B) \vee \boldsymbol{\overline{C} \vee \overline{A}} \vee (B \cdot C) \equiv $$

\framebreak 

$$\boldsymbol{(A \cdot B)} \vee \overline{C} \vee \boldsymbol{\overline{A}} \vee (B \cdot C) \equiv $$

$$[ \boldsymbol{FORMULA:} \overline{A} \vee (A \cdot B) = \overline{A} \vee B ]$$

$$B \vee \overline{C} \vee \overline{A} \vee (B \cdot C) \equiv $$


$$B \vee \overline{C} \vee \overline{A} \vee B \equiv $$

$$B \vee \overline{C} \vee \overline{A} $$

\end{frame}


\end{document}
