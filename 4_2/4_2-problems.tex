\documentclass{article}
\usepackage[margin=0.25in]{geometry}
\usepackage[utf8]{inputenc}


\usepackage[T2A]{fontenc}
\usepackage[utf8]{inputenc}
\usepackage[russian]{babel}
\usepackage{amssymb}
\usepackage{multicol}
\usepackage{amsmath}
\usepackage{tikz}

\begin{document}

\textbf{Определение}

Пусть X — произвольное множество, тогда множество $2^X = \{Y | Y \subseteq X\}$
называется булеаном множества $X$.

Примеры:

\begin{itemize}
    \item $X = \{a, b, c\}$
    
    $2^X = \{\emptyset, \{a\}, \{b\}, \{c\}, \{a, b\}, \{a, c\}, \{b, c\}, \{a, b, c\}\}$
    
    \item $X = \{\emptyset, \{ \emptyset \}\}$
    
    Замечание: $\emptyset \neq \{ \emptyset \}$
    
    $2^X = \{ \emptyset, \{\emptyset\}, \{\{ \emptyset \}\}, \{\emptyset, \{  \emptyset \}\}\}$
    
\end{itemize}

\textbf{6a}

Докажите, что 
$$2 ^ {A_1 \cap A_2 \cap ... A_n} = 2 ^ {A_1} \cap 2^{A_2} \cap ... \cap 2^{A_n}$$

\textbf{Решение:}

Воспользуемся методом математической индукции, поэтому можно доказать для двух элементов. 

Пусть у нас есть $2$ множества $A, B$. 

Элементы этих множеств можно разбить на $3$ категории 

\begin{itemize}
    \item $A \setminus (A \cap B)$
    \item $B \setminus (A \cap B)$
    \item $A \cap B$
\end{itemize}

Тогда, $2^{(A \cap B)}$ -- все подмножества, состоящие только из элементов третьей группы

Аналогичное определение и у величины $2^A \cap 2^B$ -- среди всех подмножетств множества $A$ и множетва $B$ выбраны только те, в которых есть только элементы из третьей группы. 

Значит, 
$$2^{(A \cap B)} = 2^A \cap 2^B$$

Далее, рассуждения по методу математичсекой индукции

\newpage

\textbf{6б}

Перечислите все элементы множества $2^A$, где $A = \{1, 2, \{\{1\}, 2, 3\}\}$

\textbf{Решение:}

Если ввести обозначение $c = \{\{1\}, 2, 3\}$ 

$$2^X = \{\emptyset, \{1\}, \{2\}, \{c\}, \{1, 2\}, \{1, c\}, \{2, c\}, \{1, 2, c\}\}$$

$$2^X = \{\emptyset, \{1\}, \{2\}, \{  \{\{1\}, 2, 3\} \}, \{1, 2\}, \{1, \{\{1\}, 2, 3\} \}, \{2, \{\{1\}, 2, 3\}\}, \{1, 2, \{\{1\}, 2, 3\}\}\}$$

\newpage

\textbf{Определение}

Пусть $f: X \rightarrow Y$ -- произвольное отображение.

\textbf{Образ}

$A \subseteq X$, $f(A) = \{f(x) | x \in A\}$ -- образ множества $A$ при отображении $f$. 

\textbf{Прообраз}

$B \subseteq X$, $f^{-1}(B) = \{ x \in X | f(x) \in B\}$ -- прообраз множества $A$ при отображении $f$. 

\bigskip

\textbf{7a}

Пусть $f: X \rightarrow Y$ -- произвольное отображение. $A_i \subseteq X, B_i \subseteq Y$, где $i = 1, 2, ... n$

Докажите, следующее свойство образов и прообразов. 

$$f(A_1 \cup A_2 \cup ... \cup A_n) = f(A_1) \cup f(A_2) \cup ... \cup f(A_n)$$

\textbf{Решение:}

Воспользуемся методом математической индукции 

Поэтому доказательство можно провести для двух множеств. 

Доказать, что 

$$f(A \cup B) = f(A) \cup f(B)$$

Чтобы доказать, что два множества равны между собой, докажем, что каждое из них является подмножеством другого множества

\begin{itemize}
    \item $f(A \cup B) \subseteq f(A) \cup f(B)$
    
    Без потери общности, зафиксируем $x \in A$. Тогда, результат отображения лежит в образе $f(x) \in f(A)$
    
    Значит,  $f(x) \in f(A \cup B)$, также $f(x) \in f(A) \cup f(B)$.
    
    $x$ -- принимает все значения из $(A \cup B)$
    
    \item $f(A) \cup f(B) \subseteq  f(A \cup B)$
    
    Рассуждения аналогичны. 
    
    Без потери общности, зафиксируем $x \in A$. Тогда, результат отображения лежит в образе $f(x) \in f(A)$
    
    
    
\end{itemize}

\newpage

\textbf{8}

Пусть $U$ -- универсальное множество, $S, T \subseteq U$ -- фиксированные подмножества множества $U$. Определим отображение $f: 2^U \rightarrow 2^U$, как $f(A) = T \cap (S \cup A)$. Найдите $f^{(2)}$. Выясните, чему равно $f^{n}$.

\textbf{Решение:}

Дано
$$f(A) = T \cap (S \cup A)$$

\begin{equation} 
\begin{split}
f^{(2)}(A) &= f(f(A)) \\
&=  T \cap (S \cup f(A)) \\
&= T \cap (S \cup  (T \cap (S \cup A))   ) \\
&= [use: X \cap (Y \cup Z) = (X \cap Y) \cup (X \cap Z) ] \\
&= (T \cap S) \cup  (T \cap T \cap (S \cup A))   ) \\
&= (T \cap S) \cup  (T \cap (S \cup A))   ) \\
&= [use: X \cup (Y \cap Z) = (X \cup Y) \cap (X \cup Z) ] \\
&= (T \cap S) \cup  (T \cap S) \cup (T \cap A)   \\
&= (T \cap S) \cup (T \cap A)   \\
&=  T \cap (S \cup A) \\
\end{split}
\end{equation}

Значит,  $f(f(A)) = f(A)$, $f^{(n)}(A) = f(A)$

\newpage


\textbf {Определение}

\textbf{Инъективность отображения}

Если $f(x_1) = f(x_2)$, то $x_1 = x_2$

\textbf{Сюръективность отображения}

Если $f(X) = Y$.

Если все элементы из области значения достижимы в отображении.

\textbf{Биективность отображения}

Если отображение и сюръективно, и инъективно одновременно


\bigskip

\textbf{9}

Отображения $f, g, h: 2^N \times 2 ^ N \rightarrow 2^N$ заданы следующим образом: $f(A, B) = A \cap B$, $g(A, B) = A \cup B$, $h(A, B) = A \oplus B$.

Выясните, какие из этих отображений являются инъективными, сюръективными и биективными.

Установите, какие из указанных ниже множеств не являются конечными, и перчислите элементы конечных множеств:

\begin{itemize}
    \item $f^{-1}(\emptyset)$
    \item $g^{-1}(\emptyset)$
    \item $h^{-1}(\emptyset)$ 
    \item $f^{-1}(\{1\})$
    \item $g^{-1}(\{2\})$
    \item $h^{-1}(\{3\})$
    \item $f^{-1}(\{4, 7\})$
    \item $g^{-1}(\{8, 12\})$
    \item $h^{-1}(\{5, 9\})$
\end{itemize}

\textbf{Решение}

\begin{center}
\begin{tabular}{ c c c c }
  Функция & Инъективна & Сюрьективна & Биективная \\
  $f(A, B) = A \cap B$ & - & + & -\\
  $g(A, B) = A \cup B$ & - & + & -\\
  $h(A, B) = A \oplus B$ & - & + &- \\
\end{tabular}
\end{center}


$$A \cap B$$ -- сюръективная ?????

$X$ - подмножество из области значений 

Сюръективная функция она достигает всех значений.

$A = X, B = X, A \cap B = X$

$X$ - достижимо, значит, эта функция сюрьективная



$A = X, B = X, A \cup B = X$

$A = X, B = \emptyset, A \cup B = X$

$A = X, B = \emptyset, A \oplus B = X$












$$A \cap B$$ -- инъективная ?????

Можно подобрать две пары множетсв $A, B$

$A = \{1, 2\}, B = \{1, 3\}, A \cap B = \{1\} $

$A = \{1, 10\}, B = \{1, 19\}, A \cap B = \{1\} $ 


$$A \cup B$$ -- инъективная ?????

Можно подобрать две пары множетсв $A, B$

$A = \{1, 2, 3, 4\}, B = \{1, 3\}, A \cup B = \{1, 2, 3, 4\} $

$A = \{1\}, B = \{2, 3, 4\}, A \cup B = \{1, 2, 3, 4\} $ 

$$A \oplus B$$ -- инъективная ?????

$$A \oplus B = (A \cup B ) \ (A \cap B)$$

Можно подобрать две пары множетсв $A, B$

$A = \{1, 2, 3\}, B = \{1, 5, 6\}, A \oplus B = \{2, 3, 5, 6\} $

$A = \{1\}, B = \{1, 2, 3, 5, 6\}, A \oplus B = \{2, 3, 5, 6\}  $ 




\begin{center}
\begin{tabular}{ c c }
  Функция &  \\
  $f^{-1}(\emptyset)$ &  бесконечно, $(A_1, A_2)$, где $A_1 \cup A_2 = \emptyset$\\
  $g^{-1}(\emptyset)$ &  $(\emptyset, \emptyset)$ \\
  $h^{-1}(\emptyset)$ & бесконечно, $(A_1, A_1)$ \\
  $f^{-1}(\{1\})$ & бесконечно, два множества с общим элементом $1$\\
  $g^{-1}(\{2\})$ & $(\emptyset, \{2\}), (\{2\}, \emptyset), (\{2\}, \{2\})$ \\
  $h^{-1}(\{3\})$ & бесконечно, к примеру $(A_1, A_1 \cup \{3\})$ \\
  $f^{-1}(\{4, 7\})$ &  бесконечно, два множества с общими элементами $\{4, 7\}$\\ 
  $g^{-1}(\{8, 12\})$ & $(\emptyset, \{8, 12\}), (\{8, 12\}, \emptyset), (\{8\}, \{12\}), (\{12\}, \{8\}) $  \\ 
  $h^{-1}(\{5, 9\})$ & бесконечно, к примеру $(A_1, A_1 \cup \{5, 9\})$ \\
\end{tabular}
\end{center}

\newpage

\textbf{10}

Выясните, для каких значений $n \in N$ отображение $f_n : N \cup {0} \rightarrow N$ является иньективным, сюрьективным, биективным 

$$f_n(k) = \begin{cases} n - k , & \mbox{if } k < n  \\ n + k, & \mbox{if } k \ge n \end{cases}
$$

\textbf{Решение:}

Давайте посмотрим на значения функции при $n = 1$

$$f_1(k) = \begin{cases} 1 - k , & \mbox{if } k < 1  \\ 1 + k, & \mbox{if } k \ge 1 \end{cases}
$$

\begin{tikzpicture}

\draw[step=1cm,gray,very thin] (-1.9,-1.9) grid (5.9,5.9);
\draw[thick,->] (0,0) -- (4.5,0);
\draw[thick,->] (0,0) -- (0,4.5);
\filldraw[fill=blue!40!white, draw=black] (0,1) circle (0.1);
\filldraw[fill=blue!40!white, draw=black] (1,2) circle (0.1);
\filldraw[fill=blue!40!white, draw=black] (2,3) circle (0.1);
\filldraw[fill=blue!40!white, draw=black] (3,4) circle (0.1);
\end{tikzpicture}

\begin{center}
\begin{tabular}{ c c c c }
  Функция & Инъективна & Сюрьективна & Биективная \\
  $f_1(k)$ & + & + & + \\
\end{tabular}
\end{center}


Давайте посмотрим на значения функции при $n = 2$

$$f_2(k) = \begin{cases} 2 - k , & \mbox{if } k < 2  \\ 2 + k, & \mbox{if } k \ge 2 \end{cases}
$$

\begin{tikzpicture}

\draw[step=1cm,gray,very thin] (-1.9,-1.9) grid (5.9,5.9);
\draw[thick,->] (0,0) -- (4.5,0);
\draw[thick,->] (0,0) -- (0,4.5);
\filldraw[fill=blue!40!white, draw=black] (0,2) circle (0.1);
\filldraw[fill=blue!40!white, draw=black] (1,1) circle (0.1);
\filldraw[fill=blue!40!white, draw=black] (2,4) circle (0.1);
\filldraw[fill=blue!40!white, draw=black] (3,5) circle (0.1);
\end{tikzpicture}

\begin{center}
\begin{tabular}{ c c c c }
  Функция & Инъективна & Сюрьективна & Биективная \\
  $f_2(k)$ & + & - & - \\
\end{tabular}
\end{center}


Давайте посмотрим на значения функции при $n = 3$

$$f_3(k) = \begin{cases} 3 - k , & \mbox{if } k < 3  \\ 3 + k, & \mbox{if } k \ge 3 \end{cases}
$$

\begin{tikzpicture}

\draw[step=1cm,gray,very thin] (-1.9,-1.9) grid (7.9,10.9);
\draw[thick,->] (0,0) -- (6.5,0);
\draw[thick,->] (0,0) -- (0,9.5);
\filldraw[fill=blue!40!white, draw=black] (0,3) circle (0.1);
\filldraw[fill=blue!40!white, draw=black] (1,2) circle (0.1);
\filldraw[fill=blue!40!white, draw=black] (2,1) circle (0.1);
\filldraw[fill=blue!40!white, draw=black] (3,6) circle (0.1);
\filldraw[fill=blue!40!white, draw=black] (4,7) circle (0.1);
\filldraw[fill=blue!40!white, draw=black] (5,8) circle (0.1);
\end{tikzpicture}

\begin{center}
\begin{tabular}{ c c c c }
  Функция & Инъективна & Сюрьективна & Биективная \\
  $f_3(k)$ & + & - & - \\
\end{tabular}
\end{center}


Видно, что график состоит из двух частей. Тогда при $n > 1$

\begin{center}
\begin{tabular}{ c c c c }
  Функция & Инъективна & Сюрьективна & Биективная \\
  $f_n(k)$ & + & - & - \\
\end{tabular}
\end{center}


\end{document}
