%% This Beamer template is based on the one found here: https://github.com/sanhacheong/stanford-beamer-presentation, and edited to be used for Stanford ARM Lab

\documentclass[10pt]{beamer}
%\mode<presentation>{}

\usepackage{media9}
\usepackage{amssymb,amsmath,amsthm,enumerate}
\usepackage[utf8]{inputenc}
\usepackage{array}
\usepackage[parfill]{parskip}
\usepackage{graphicx}
\usepackage{caption}
\usepackage{subcaption}
\usepackage{bm}
\usepackage{amsfonts,amscd}
\usepackage[]{units}
\usepackage{listings}
\usepackage{multicol}
\usepackage{multirow}
\usepackage{tcolorbox}
\usepackage{physics}
\usepackage[T2A]{fontenc}
\usepackage[utf8]{inputenc}
\usepackage[russian]{babel}

% Enable colored hyperlinks
\hypersetup{colorlinks=true}

% The following three lines are for crossmarks & checkmarks
\usepackage{pifont}% http://ctan.org/pkg/pifont
\newcommand{\cmark}{\ding{51}}%
\newcommand{\xmark}{\ding{55}}%

% Numbered captions of tables, pictures, etc.
\setbeamertemplate{caption}[numbered]

%\usepackage[superscript,biblabel]{cite}
\usepackage{algorithm2e}
\renewcommand{\thealgocf}{}

% Bibliography settings
\usepackage[style=ieee]{biblatex}
\setbeamertemplate{bibliography item}{\insertbiblabel}
\addbibresource{references.bib}

% Glossary entries
\usepackage[acronym]{glossaries}
\newacronym{ML}{ML}{machine learning}
\newacronym{HRI}{HRI}{human-robot interactions}
\newacronym{RNN}{RNN}{Recurrent Neural Network}
\newacronym{LSTM}{LSTM}{Long Short-Term Memory}


\theoremstyle{remark}
\newtheorem*{remark}{Remark}
\theoremstyle{definition}

\newcommand{\empy}[1]{{\color{darkorange}\emph{#1}}}
\newcommand{\empr}[1]{{\color{cardinalred}\emph{#1}}}
\newcommand{\examplebox}[2]{
\begin{tcolorbox}[colframe=darkcardinal,colback=boxgray,title=#1]
#2
\end{tcolorbox}}

\usetheme{Stanford} 
\input{./style_files_stanford/my_beamer_defs.sty}


% commands to relax beamer and subfig conflicts
% see here: https://tex.stackexchange.com/questions/426088/texlive-pretest-2018-beamer-and-subfig-collide
\makeatletter
\let\@@magyar@captionfix\relax
\makeatother

\title[Дискретная математика]{Вводная}
%\subtitle{Subtitle Of Presentation}

%\beamertemplatenavigationsymbolsempty

\begin{document}


\author[BSU]{
	\begin{tabular}{c} 
	\Large
	\\
    \footnotesize \href{mailto:anastasiya.zhyrkevich@yandex.ru}{anastasiya.zhyrkevich@yandex.ru}
\end{tabular}
\vspace{-4ex}}

\institute{
	\vskip 5pt
	\vskip 5pt
	Дискретная математика\\
	Семинар 1\\
	\vskip 3pt
}

\date{09 February, 2020}
% \date{\today}

\begin{noheadline}
\begin{frame}\maketitle\end{frame}
\end{noheadline}

\setbeamertemplate{itemize items}[default]
\setbeamertemplate{itemize subitem}[circle]

\begin{frame}
	\frametitle{Краткое содержание} % Table of contents slide, comment this block out to remove it
	\tableofcontents % Throughout your presentation, if you choose to use \section{} and \subsection{} commands, these will automatically be printed on this slide as an overview of your presentation
\end{frame}

\section{Структура курса}
\begin{frame}[allowframebreaks]
\frametitle{Дискретная математика}

\begin{enumerate}
    \item Курс состоит из 16 семинарских занятий
    \item Зачет в конце курса
    \item 2 теста на irunner, 2 контрольные работы 
    \item Отметка за курс - среднее письменных работ
\end{enumerate}

\framebreak

\begin{enumerate}
    \item Семинары в формате разбора задач
    \item Требования к письменным работам
    \item Начало курса может показаться простым
    \item Задавайте вопросы

\end{enumerate}

\framebreak 

Самая полезная ссылка

\href{https://acm.bsu.by/wiki/DM2020}{https://acm.bsu.by/wiki/DM2020}

Все данные, все условия задач.

Полезные ссылки 
\begin{enumerate}
    \item \href{https://github.com/Anastasiya-Zhyrkevich/DiscreteMath-solutions}{github} с полными данными семинаров
    \item \href{https://drive.google.com/drive/folders/1LO6ifRf9qDqoghwRaMIpPo1crvZeAjjn?usp=sharing}{Диск} с данными семинаров
\end{enumerate}

\framebreak

\end{frame}


\section{Знакомство}
\begin{frame}[allowframebreaks]
\frametitle{Знакомство}

Жиркевич Анастасия Борисовна 

Телеграм @anastzhyr 

+375296601770

Anastasiya.Zhyrkevich@gmail.com

\framebreak 

Игра

\end{frame}


\section{Введение в математическую логику}
\begin{frame}[allowframebreaks]
\frametitle{Введение в математическую логику}

Материал лекции: 
\begin{enumerate}
    \item Понятие простого высказывания True, False
    \item Логические связки:
\begin{itemize}
    \item НЕ  (отрицание) -- $\overline{A}$
    \item И (конъюнкция) -- $\wedge$ $\cdot$
    \item ИЛИ (дизъюнкция) -- $\vee$
    \item СЛЕДУЕТ (импликация) -- $\to$
\end{itemize}
\item Таблица истинности формулы
\end{enumerate}

\end{frame}

\section{Решение задачи 1}
\begin{frame}[allowframebreaks]
\frametitle{Решение задачи 1}

Импликация. 

$$ A \to B $$

Таблица истинности 

\begin{center}
\begin{tabular}{ c c c }
 A & B & A \to B \\ 
 0 & 0 & 1 \\  
 0 & 1 & 1 \\ 
 1 & 0 & 0 \\ 
 1 & 1 & 1 \\
\end{tabular}
\end{center}

\framebreak

$$A \to B $$

Аналоги в русском языке: 
\begin{itemize}
    \item Если $A$, то $B$
    \item $B$ в том случае, если $A$
    \item При $A$ будет $B$
    \item Из $A$ следует $B$
    \item В случае $A$ произойдёт $B$
    \item $B$, так как $A$
    \item $B$, потому что $A$
    \item $A$ — достаточное условие для $B$
    \item $B$ — необходимое условие для $A$
\end{itemize}

\framebreak 

Задание: 

Выделив условие и заключение теоремы, сформулируйте ее посредством связки "Если ..., то ... "

\framebreak 

Задание: 

Выделив условие и заключение теоремы, сформулируйте ее посредством связки "Если ..., то ... "

1. Для того, чтобы функция была дифференцируема в некоторой точке, необходимо, чтобы она была непрерывна в этой точке


\framebreak 

Задание: 

Выделив условие и заключение теоремы, сформулируйте ее посредством связки "Если ..., то ... "

1. Для того, чтобы функция была дифференцируема в некоторой точке, необходимо, чтобы она была непрерывна в этой точке

\textit{Ответ:} 

Если функция была дифференцируема в некоторой точке, то она непрерывна в этой точке

\framebreak 

Задание: 

Выделив условие и заключение теоремы, сформулируйте ее посредством связки "Если ..., то ... "

2. Необходимым свойством прямоугольника является равенство его диагоналей. 


\framebreak 

Задание: 

Выделив условие и заключение теоремы, сформулируйте ее посредством связки "Если ..., то ... "

2. Необходимым свойством прямоугольника является равенство его диагоналей. 

\textit{Ответ:} 

Если фигура прямоугольник, то у него равны стороны.


\framebreak 

Задание: 

Выделив условие и заключение теоремы, сформулируйте ее посредством связки "Если ..., то ... "

3. Для делимости многочлена $f(x)$ на линейный двучлен $x - a$ достаточно, чтобы исло $a$ было корнем этого многочлена.


\framebreak 

Задание: 

Выделив условие и заключение теоремы, сформулируйте ее посредством связки "Если ..., то ... "

3. Для делимости многочлена $f(x)$ на линейный двучлен $x - a$ достаточно, чтобы число $a$ было корнем этого ногочлена.

\textit{Ответ:} 

Если число $a$ корень многочлена, то многочлен $f(x)$ делится на линейный двучлен $x - a$


\framebreak 

Задание: 

Выделив условие и заключение теоремы, сформулируйте ее посредством связки "Если ..., то ... "

4. На 5 делятся те целые числа, которые оканчиваются цифрой 0 или цифрой 5


\framebreak 

Задание: 

Выделив условие и заключение теоремы, сформулируйте ее посредством связки "Если ..., то ... "

4. На 5 делятся те целые числа, которые оканчиваются цифрой 0 или цифрой 5

\textit{Ответ:} 

Если число оканчивается на 0 или 5, то число делится на 5.

\framebreak 

Задание: 

Выделив условие и заключение теоремы, сформулируйте ее посредством связки "Если ..., то ... "

5. Две прямые на плоскости тогда параллельны, когда они перпендикулярны одной и той же прямой.


\framebreak 

Задание: 

Выделив условие и заключение теоремы, сформулируйте ее посредством связки "Если ..., то ... "

5. Две прямые на плоскости тогда параллельны, когда они перпендикулярны одной и той же прямой.

\textit{Ответ:} 

Если две прямые перпендикулярны одной и той же прямой, то они параллельны.

\framebreak 

Задание: 

Выделив условие и заключение теоремы, сформулируйте ее посредством связки "Если ..., то ... "

6. Комплексные числа равны, только если равны соответсвенно их действительные и мнимые части.


\framebreak 

Задание: 

Выделив условие и заключение теоремы, сформулируйте ее посредством связки "Если ..., то ... "

6. Комплексные числа равны, только если равны соответсвенно их действительные и мнимые части.

\textit{Ответ:} 

Если равны соответсвенно действительные и мнимые части комплексных чисел, то числа равны.

\framebreak 

Задание: 

Выделив условие и заключение теоремы, сформулируйте ее посредством связки "Если ..., то ... "

7. Всякое квадратное уравнение с действительными коэффициентами имеет не более двух действительных корней.


\framebreak 

Задание: 

Выделив условие и заключение теоремы, сформулируйте ее посредством связки "Если ..., то ... "

7. Всякое квадратное уравнение с действительными коэффициентами имеет не более двух действительных корней.

\textit{Ответ:} 

Если уравнение с действительными коэффииентами, то оно имеет не более двух действительных корней.


\framebreak 

Задание: 

Выделив условие и заключение теоремы, сформулируйте ее посредством связки "Если ..., то ... "

8. Из того, что четырехугольник -- ромб, следует, что каждая ихз его диагоналей служит его осью симметрии.


\framebreak 

Задание: 

Выделив условие и заключение теоремы, сформулируйте ее посредством связки "Если ..., то ... "

8. Из того, что четырехугольник -- ромб, следует, что каждая ихз его диагоналей служит его осью симметрии.

\textit{Ответ:} 

Если четырехугольник -- ромб, то  каждая ихз его диагоналей служит его осью симметрии.

\framebreak 

Задание: 

Выделив условие и заключение теоремы, сформулируйте ее посредством связки "Если ..., то ... "

9. Четность суммы есть необходимое условие четности каждого слагаемого.


\framebreak 

Задание: 

Выделив условие и заключение теоремы, сформулируйте ее посредством связки "Если ..., то ... "

9. Четность суммы есть необходимое условие четности каждого слагаемого.

\textit{Ответ:} 

Если каждое слагаемое четно, то сумма - четна.

\framebreak 

Задание: 

Выделив условие и заключение теоремы, сформулируйте ее посредством связки "Если ..., то ... "

10. Равенство треугольников есть достаточное условие их равновеликости.


\framebreak 

Задание: 

Выделив условие и заключение теоремы, сформулируйте ее посредством связки "Если ..., то ... "

10. Равенство треугольников есть достаточное условие их равновеликости.

\textit{Ответ:} 

Если треугольники равны, то они равновелики.

\framebreak 

Задание: 

Выделив условие и заключение теоремы, сформулируйте ее посредством связки "Если ..., то ... "

11. Для делимости произведения на некоторое число достаточно,  чтобы по меньшей мере один из сомножителей делился на это число.


\framebreak 

Задание: 

Выделив условие и заключение теоремы, сформулируйте ее посредством связки "Если ..., то ... "

11. Для делимости произведения на некоторое число достаточно,  чтобы по меньшей мере один из сомножителей делился на это число.

\textit{Ответ:} 

Если хоть один из сомножителей делится на число, то и произведение делится на это же число.


\end{frame}


\section{Составление таблицы истинности}
\begin{frame}[allowframebreaks]
\frametitle{Составление таблицы истинности. Решение задачи 3}

Составим таблицу истинности 

$$(A \to B) \vee (A \to (A \cdot B)) $$

\framebreak 

$$(A \to B) \vee (A \to (A \cdot B)) $$

Нумеруем последовательность операций по приоритету.
\begin{enumerate}
    \item $A \to B$
    \item $A \cdot B$
    \item $A \to (A \cdot B)$
    \item $(A \to B) \vee (A \to (A \cdot B)) = \alpha$
\end{enumerate}

\framebreak

\begin{center}
\begin{tabular}{ c c c c c c }
 A & B & (A \to B) & . & . & . \\ 
 0 & 0 & 1 & . & . & . \\  
 0 & 1 & 1 & . & . & . \\ 
 1 & 0 & 0 & . & . & . \\ 
 1 & 1 & 1 & . & . & . \\
\end{tabular}
\end{center}

\framebreak

\begin{center}
\begin{tabular}{ c c c c c c }
 A & B & (A \to B) & A \cdot B & . & . \\ 
 0 & 0 & 1 & 0 & . & . \\  
 0 & 1 & 1 & 0 & . & . \\ 
 1 & 0 & 0 & 0 & . & . \\ 
 1 & 1 & 1 & 1 & . & . \\
\end{tabular}
\end{center}

\framebreak

\begin{center}
\begin{tabular}{ c c c c c c }
 A & B & $(A \to B)$ & $A \cdot B$ & $A \to (A \cdot B)$ & . \\ 
 0 & 0 & 1 & 0 & 1 & . \\  
 0 & 1 & 1 & 0 & 1 & . \\ 
 1 & 0 & 0 & 0 & 0 & . \\ 
 1 & 1 & 1 & 1 & 1 & . \\
\end{tabular}
\end{center}

\framebreak

$$(A \to B) \vee (A \to (A \cdot B)) = \alpha$$

\begin{center}
\begin{tabular}{ c c c c c c }
 A & B & (A \to B) & A \cdot B & A \to (A \cdot B) & \alpha \\ 
 0 & 0 & 1 & 0 & 1 & 1 \\  
 0 & 1 & 1 & 0 & 1 & 1 \\ 
 1 & 0 & 0 & 0 & 0 & 0 \\ 
 1 & 1 & 1 & 1 & 1 & 1 \\
\end{tabular}
\end{center}
\end{frame}

\section{Составление таблицы истинности}
\begin{frame}[allowframebreaks]
\frametitle{Решение задачи 3}

\begin{itemize}
    \item $((A \thicksim B) \to \overline{C}) \cdot (A \vee C)$.

\item $(((\overline{A \vee B}) \cdot \overline{C}) \to \overline{B}) \thicksim A$.

\item $((\overline{A} \cdot \overline{B}) \to (\overline{\overline{B} \to \overline{A}}))
\cdot ((A \vee B) \thicksim C)$.
\end{itemize}

\end{frame}


\end{document}
