\documentclass{article}
\usepackage[margin=0.25in]{geometry}
\usepackage[utf8]{inputenc}


\usepackage[T2A]{fontenc}
\usepackage[utf8]{inputenc}
\usepackage[russian]{babel}
\usepackage{amssymb}
\usepackage{multicol}
\usepackage{amsmath}
\usepackage{tikz}

\begin{document}

\textbf{Общие комментарии}

Все диаграммы не являются способом доказательства, а лишь использованы для понимания и наглядности. 

\textbf{1.}

\textbf{(a)}

Докажите следующее тождество, используя определение равенства множеств:

$$A \cup (B \cap C) = (A \cup B) \cap (A \cup C)$$

\textbf{Решение:}
Докажем тождество, используя формулы логики высказывания и определения множеств. 

\begin{equation} 
\begin{split}
A \cup (B \cap C) & = \{x | x \in A \cup (B \cap C) \} \\
 & = \{x | (x \in A) \vee (x \in (B \cap C)) \} \\
 & = \{x | (x \in A) \vee ( (x \in B) \wedge (x \in C)) \} \\
 & = [ use formula: X \vee (Y \wedge Z) = (X \vee Y) \wedge (X \vee Z)]  \\ 
 & = \{x | ((x \in A) \vee (x \in B)) \wedge ((x \in A) \vee (x \in C)) \} \\
 & = \{x | ((x \in A) \vee (x \in B)) \wedge ((x \in A) \vee (x \in C)) \} \\
 & = (A \cup B) \cap (A \cup C) \\
\end{split}
\end{equation}

Приведем для наглядности диаграммы. Диаграммы не являются способом доказательства, а лишь использованы для понимания и наглядности. 

\begin{multicols}{2}
[ Левая часть равенства $A \cup (B \cap C)$ ]

\def\firstcircle{(0,0) circle (1.5cm)}
\def\secondcircle{(0:2cm) circle (1.5cm)}
\def\thirdcircle{(1, 2) circle (1.5cm)}

\colorlet{circle edge}{blue!50}
\colorlet{circle area}{blue!20}

\tikzset{filled/.style={fill=circle area, draw=circle edge, thick},
    outline/.style={draw=circle edge, thick}}

\begin{tikzpicture}
    \begin{scope}
        \clip \secondcircle;
        \fill[filled] \thirdcircle;
    \end{scope}
    
    \draw[outline] \firstcircle node {$A$};
    \draw[outline] \secondcircle node {$B$};
    \draw[outline] \thirdcircle node {$C$};
    
    \node[anchor=south] at (current bounding box.north) {$B \cap C$};
\end{tikzpicture}

\begin{tikzpicture}
    \begin{scope}
        \clip \secondcircle;
        \fill[filled] \thirdcircle;
    \end{scope}
    
    \begin{scope}
        \fill[filled] \firstcircle;
    \end{scope}
    
    \draw[outline] \firstcircle node {$A$};
    \draw[outline] \secondcircle node {$B$};
    \draw[outline] \thirdcircle node {$C$};
    
    \node[anchor=south] at (current bounding box.north) {$A \cup (B \cap C)$};
\end{tikzpicture}

\end{multicols}

\begin{multicols}{3}
[ Правая часть равенства $(A \cup B) \cap (A \cup C)$ ]

\def\firstcircle{(0,0) circle (1.5cm)}
\def\secondcircle{(0:2cm) circle (1.5cm)}
\def\thirdcircle{(1, 2) circle (1.5cm)}

\colorlet{circle edge}{blue!50}
\colorlet{circle area}{blue!20}

\tikzset{filled/.style={fill=circle area, draw=circle edge, thick},
    outline/.style={draw=circle edge, thick}}

\begin{tikzpicture}
    \begin{scope}
        \fill[filled] \firstcircle;
    \end{scope}
    
    \begin{scope}
        \fill[filled] \secondcircle;
    \end{scope}
    
    \draw[outline] \firstcircle node {$A$};
    \draw[outline] \secondcircle node {$B$};
    \draw[outline] \thirdcircle node {$C$};
    
    \node[anchor=south] at (current bounding box.north) {$A \cup B$};
\end{tikzpicture}

\begin{tikzpicture}
    \begin{scope}
        \fill[filled] \firstcircle;
    \end{scope}
    
    \begin{scope}
        \fill[filled] \thirdcircle;
    \end{scope}
    
    \draw[outline] \firstcircle node {$A$};
    \draw[outline] \secondcircle node {$B$};
    \draw[outline] \thirdcircle node {$C$};
    
    \node[anchor=south] at (current bounding box.north) {$A \cup C$};
\end{tikzpicture}

\begin{tikzpicture}
    \begin{scope}
        \clip \secondcircle;
        \fill[filled] \thirdcircle;
    \end{scope}
    
    \begin{scope}
        \fill[filled] \firstcircle;
    \end{scope}
    
    \draw[outline] \firstcircle node {$A$};
    \draw[outline] \secondcircle node {$B$};
    \draw[outline] \thirdcircle node {$C$};
    
    \node[anchor=south] at (current bounding box.north) {$(A \cup B) \cap (A \cup C)$};
\end{tikzpicture}

\end{multicols}

\newpage

\textbf{(б)}

Докажите следующее тождество, используя определение равенства множеств:

$$A \setminus (B \cup C) = (A \setminus B) \cap (A \setminus C)$$

\textbf{Решение:}

Докажем тождество, используя формулы логики высказывания и определения множеств. 

\begin{equation} 
\begin{split}
A \setminus (B \cup C) & = \{x | x \in A \setminus (B \cup C) \} \\
 & = \{x | (x \in A) \wedge (x \notin (B \cup C)) \} \\
 & = \{x | (x \in A) \wedge ( (x \notin B) \vee (x \notin C)) \} \\
 & = [ use formula: X \wedge (Y \vee Z) = (X \wedge Y) \vee (X \wedge Z)]  \\ 
 & = \{x | ((x \in A) \wedge (x \notin B)) \vee ((x \in A) \wedge (x \notin C)) \} \\
 & = (A \setminus B) \cap (A \setminus C) \\
\end{split}
\end{equation}


Приведем для наглядности диаграммы. 

\begin{multicols}{2}
[ Левая часть равенства $A \setminus (B \cup C)$ ]

\def\firstcircle{(0,0) circle (1.5cm)}
\def\secondcircle{(0:2cm) circle (1.5cm)}
\def\thirdcircle{(1, 2) circle (1.5cm)}

\colorlet{circle edge}{blue!50}
\colorlet{circle area}{blue!20}

\tikzset{filled/.style={fill=circle area, draw=circle edge, thick},
    outline/.style={draw=circle edge, thick}}

\begin{tikzpicture}
    \begin{scope}
        \fill[filled] \thirdcircle;
    \end{scope}
    
    \begin{scope}
        \fill[filled] \secondcircle;
    \end{scope}
    
    \draw[outline] \firstcircle node {$A$};
    \draw[outline] \secondcircle node {$B$};
    \draw[outline] \thirdcircle node {$C$};
    
    \node[anchor=south] at (current bounding box.north) {$B \cup C$};
\end{tikzpicture}

\begin{tikzpicture}
    \begin{scope}[even odd rule]
        \clip \thirdcircle (-4,-4) rectangle (4,4);
        \clip \secondcircle (-4,-4) rectangle (4,4);
      
        \fill[filled] \firstcircle;
    \end{scope}
    
    \draw[outline] \firstcircle node {$A$};
    \draw[outline] \secondcircle node {$B$};
    \draw[outline] \thirdcircle node {$C$};
    
    \node[anchor=south] at (current bounding box.north) { $A \setminus (B \cup C)$ };
\end{tikzpicture}

\end{multicols}

\begin{multicols}{3}
[ Правая часть равенства $(A \setminus B) \cap (A  \setminus C)$ ]

\def\firstcircle{(0,0) circle (1.5cm)}
\def\secondcircle{(0:2cm) circle (1.5cm)}
\def\thirdcircle{(1, 2) circle (1.5cm)}

\colorlet{circle edge}{blue!50}
\colorlet{circle area}{blue!20}

\tikzset{filled/.style={fill=circle area, draw=circle edge, thick},
    outline/.style={draw=circle edge, thick}}

\begin{tikzpicture}
    \begin{scope}[even odd rule]
        \clip \secondcircle (-2,-2) rectangle (4,4);
      
        \fill[filled] \firstcircle;
    \end{scope}
    
    \draw[outline] \firstcircle node {$A$};
    \draw[outline] \secondcircle node {$B$};
    \draw[outline] \thirdcircle node {$C$};
    
    \node[anchor=south] at (current bounding box.north) {$A \setminus B$};
\end{tikzpicture}

\begin{tikzpicture}
    \begin{scope}[even odd rule]
        \clip \thirdcircle (-2, -2) rectangle (4,4);
      
        \fill[filled] \firstcircle;
    \end{scope}
    
    \draw[outline] \firstcircle node {$A$};
    \draw[outline] \secondcircle node {$B$};
    \draw[outline] \thirdcircle node {$C$};
    
    \node[anchor=south] at (current bounding box.north) {$A \setminus C$};
\end{tikzpicture}

\begin{tikzpicture}
    \begin{scope}[even odd rule]
        \clip \thirdcircle (-2,-2) rectangle (4,4);
        \clip \secondcircle (-2,-2) rectangle (4,4);
      
        \fill[filled] \firstcircle;
    \end{scope}
    
    \draw[outline] \firstcircle node {$A$};
    \draw[outline] \secondcircle node {$B$};
    \draw[outline] \thirdcircle node {$C$};
    
    \node[anchor=south] at (current bounding box.north) { $(A \setminus B) \cap (A  \setminus C)$ };
\end{tikzpicture}

\end{multicols}


\newpage

\textbf{2} 

\textbf{(a)}

Докажите следующее тождества, используя равносильные преобразования:

$$ (A \cap B) \cup (C \cap D) = (A \cup C) \cap (A \cup D) \cap (B \cup C) \cap (B \cup D)$$

\textbf{Решение:}
Необходимо доказать следующее тождество

$$ (A \cap B) \cup (C \cap D) = (A \cup C) \cap (A \cup D) \cap (B \cup C) \cap (B \cup D)$$

\begin{equation} 
\begin{split}
(A \cap B) \cup (C \cap D) & =  \\
 & = [X \cup (Y \cap Z) = (X \cup Y) \cap (X \cup Z)] \\
 & = ((A \cap B) \cup C) \cap ((A \cap B) \cup D) \\
 & = [X \cap (Y \cup Z) = (X \cap Y) \cup (X \cap Z)] \\
 & = ((A \cup C) \cap (B \cup C)) \cap ((A \cup D) \cap (B \cup D)) \\
 & = (A \cup C) \cap (B \cup C) \cap (A \cup D) \cap (B \cup D) \\
\end{split}
\end{equation}

\textbf{(б)}

Докажите следующее тождества, используя равносильные преобразования:

$$ A \oplus B \oplus (A \cap B) = A \cup B$$

\textbf{Решение:}
Необходимо доказать следующее тождество
$$ A \oplus B \oplus (A \cap B) = A \cup B$$

\begin{equation} 
\begin{split}
A \oplus B \oplus (A \cap B) & =  \\
 & = [A \oplus B = (A \setminus B) \cup (B \setminus A)] \\
 & = ((A \setminus B) \cup (B \setminus A)) \oplus (A \cap B) \\
 & = [X \oplus Y = (X \cup Y) \setminus (X \cap Y)] \\
 & = ((A \setminus B) \cup (B \setminus A) \cup (A \cap B)) \setminus (((A \setminus B) \cup (B \setminus A)) \cap (A \cap B)) \\
 & = [(X \setminus Y) \cup (X \cap Y) = X ] \\
 & = (A \cup (B \setminus A)) \setminus (((A \setminus B) \cup (B \setminus A)) \cap (A \cap B)) \\
 & = [A \cup (B \setminus A) =  A \cup B] \\
 & = (A \cup B) \setminus (((A \setminus B) \cup (B \setminus A)) \cap (A \cap B)) \\
 & = [((A \setminus B) \cup (B \setminus A)) \cap (A \cap B) = \emptyset ] \\
 & = (A \cup B) \setminus \emptyset \\
 & = (A \cup B) \\
\end{split}
\end{equation}

Для большего понимания и наглядности, приведем диаграммы

\begin{multicols}{2}
[ Левая часть равенства $A \oplus B \oplus (A \cap B)$ ]

\def\firstcircle{(0,0) circle (1.5cm)}
\def\secondcircle{(0:2cm) circle (1.5cm)}

\colorlet{circle edge}{blue!50}
\colorlet{circle area}{blue!20}

\tikzset{filled/.style={fill=circle area, draw=circle edge, thick},
    outline/.style={draw=circle edge, thick}}

\begin{tikzpicture}
    \begin{scope}[even odd rule]
        \clip \secondcircle (-2,-2) rectangle (4,4);
        \fill[filled] \firstcircle;
    \end{scope}
    
    \begin{scope}[even odd rule]
        \clip \firstcircle (-2,-2) rectangle (4,4);
        \fill[filled] \secondcircle;
    \end{scope}
    
    \draw[outline] \firstcircle node {$A$};
    \draw[outline] \secondcircle node {$B$};
    
    \node[anchor=south] at (current bounding box.north) {$A \oplus B$};
\end{tikzpicture}

\begin{tikzpicture}
    \begin{scope}
        \clip \secondcircle;
      
        \fill[filled] \firstcircle;
    \end{scope}
    
    \draw[outline] \firstcircle node {$A$};
    \draw[outline] \secondcircle node {$B$};
    
    \node[anchor=south] at (current bounding box.north) {$A \cap B$};
\end{tikzpicture}

\end{multicols}

\newpage

\begin{multicols}{3}
[  ]

\def\firstcircle{(0,0) circle (1.5cm)}
\def\secondcircle{(0:2cm) circle (1.5cm)}

\colorlet{circle edge}{blue!50}
\colorlet{circle area}{blue!20}

\tikzset{filled/.style={fill=circle area, draw=circle edge, thick},
    outline/.style={draw=circle edge, thick}}

\begin{tikzpicture}
    \begin{scope}[even odd rule]
        \fill[filled] \firstcircle;
        \fill[filled] \secondcircle;
    \end{scope}
    
    \draw[outline] \firstcircle node {$A$};
    \draw[outline] \secondcircle node {$B$};
    
    \node[anchor=south] at (current bounding box.north) {$(A \oplus B) \cup (A \cap B)$};
\end{tikzpicture}

\begin{tikzpicture}
    \begin{scope}
    \end{scope}
    
    \draw[outline] \firstcircle node {$A$};
    \draw[outline] \secondcircle node {$B$};
    
    \node[anchor=south] at (current bounding box.north) {$(A \oplus B) \cap (A \cap B)$};
\end{tikzpicture}
\end{multicols}

\begin{tikzpicture}
    \def\firstcircle{(0,0) circle (1.5cm)}
    \def\secondcircle{(0:2cm) circle (1.5cm)}
    
    \colorlet{circle edge}{blue!50}
    \colorlet{circle area}{blue!20}
    
    \tikzset{filled/.style={fill=circle area, draw=circle edge, thick},
        outline/.style={draw=circle edge, thick}}

    \begin{scope}
    \fill[filled] \firstcircle;
    \fill[filled] \secondcircle;
    \end{scope}
    
    \draw[outline] \firstcircle node {$A$};
    \draw[outline] \secondcircle node {$B$};
    
    \node[anchor=south] at (current bounding box.north) {$(A \oplus B) \oplus (A \cap B) = ((A \oplus B) \cup (A \cap B) ) \setminus  ((A \oplus B) \cap (A \cap B) )$};
\end{tikzpicture}

\textbf {3}

\textbf {(a)}

Докажите следующие утверждение:

$$ (A \cup B) \subseteq C \iff (A \subseteq C) \wedge (B \subseteq C) $$

\textbf{Решение:}
Необходимо доказать следующее утверждение

$$ (A \cup B) \subseteq C \iff (A \subseteq C) \wedge (B \subseteq C) $$

\textbf{Способ 1} 

Докажем с помощью рассуждений

$$(A \cup B) = (A \setminus B) \cup (B \setminus A) \cup (A \cap B)$$

Другими словами, все элементы в $(A \cup B)$ можно разделить на 3 группы : 
\begin{itemize}
    \item те, что входят в $A$, но не входят в $B$ ($A \setminus B$)
    \item те, что входят в $B$, но не входят в $A$ ($B \setminus A$)
    \item те, что входят в $A$ и в $B$ ($A \cap B$)
\end{itemize}
 
Элементы из каждой категории лежат в $C$ (по определению понятия подмножества) 

$$A = (A \setminus B ) \cup (A \cap B)$$
$$B = (B \setminus A ) \cup (A \cap B)$$

А значит, все элементы из $A$ лежат и в $C$. Аналогично, все элементы из $B$ лежат и в $C$

Что и требовалось.

\textbf{Способ 2} 

Введем следующие обозначения: 

$$ (x \in A) = A_0, x \in B = B_0, x \in C = C_0$$

$A_0, B_0, C_0$ - переменные логики высказываний ($True False$). 

Тогда условие переписывается в форме:
$$((A_0 \vee B_0) \rightarrow C_0) \thicksim ((A_0 \rightarrow C_0) \wedge (B_0 \rightarrow C_0)) $$

\begin{itemize}
    \item Понятие подмножество ($\subseteq$) заменяется на импликацию ($\rightarrow$)
    \item Понятие обьединения ($\cup$) заменяется на дизъюнкицию ($\vee$)
    \item Понятие пересечения ($\cap$) заменяется на конъюнкицию ($\wedge$)
\end{itemize}

Преобразуем полученное выражение
$$((A_0 \vee B_0) \rightarrow C_0) \thicksim ((A_0 \rightarrow C_0) \wedge (B_0 \rightarrow C_0)) $$

Левая часть:

$$(A_0 \vee B_0) \rightarrow C_0 = \overline{A_0 \vee B_0} \vee C_0 = (\overline{A_0} \wedge \overline{B_0}) \vee C_0 = (\overline{A_0} \vee C_0) \wedge (\overline{B_0} \vee C_0)$$

Правая часть:

$$(A_0 \rightarrow C_0) \wedge (B_0 \rightarrow C_0) = (\overline{A_0} \vee C_0) \wedge (\overline{B_0} \vee C_0)$$

Получается, что данная в условии равносильность -- тавтология, т.к. левая и правая части одинаковые 

$$((A_0 \vee B_0) \rightarrow C_0) \thicksim ((A_0 \rightarrow C_0) \wedge (B_0 \rightarrow C_0)) = True $$

\textbf{(б)}
Докажите следующие утверждение:

$$ A \subseteq (B \cup C) \iff (A \subseteq B) \wedge (A \subseteq C) $$

\textbf{Решение:}


Введем следующие обозначения: 

$$ (x \in A) = A_0, x \in B = B_0, x \in C = C_0$$

$A_0, B_0, C_0$ - переменные логики высказываний ($True False$). 

Тогда условие переписывается в форме:
$$(A_0 \rightarrow (B_0 \wedge C_0)) \thicksim ((A_0 \rightarrow B_0) \wedge (A_0 \rightarrow C_0)) $$

Преобразуем:

Левая часть
$$A_0 \rightarrow (B_0 \wedge C_0) = \overline{A_0} \vee (B_0 \wedge C_0) = (\overline{A_0} \vee B_0) \wedge (\overline{A_0} \vee C_0) $$

Правая часть
$$(A_0 \rightarrow B_0) \wedge (A_0 \rightarrow C_0) = (\overline{A_0} \vee B_0) \wedge (\overline{A_0} \vee C_0) $$

Получаем, что данная в условии равносильность -- тавтология, т.к. левая и правая части одинаковые 

\textbf{4}

\textbf{(a)}

Решите следующие системы уравнений:

$$\begin{cases} A \cap X = B \\ A \cup X = C \\ B \subseteq A \subseteq C \end{cases}$$

\textbf{Решение:}
Необходимо решить систему уравнений 

$$\begin{cases} A \cap X = B \\ A \cup X = C \\ B \subseteq A \subseteq C \end{cases}$$

\textbf{Способ 1 (сложный, но точный)}

Введем следующие обозначения: 

$$ (x \in A) = A_0, x \in B = B_0, x \in C = C_0, x \in X = X_0$$

Перепишем первое уравнение из системы

$$ A \cap X = B  \iff A \cap X \subseteq B , B \subseteq A \cap X$$

Первую часть можно преобразовать, (используя обозначения) 
$$A \cap X \subseteq B = X \subseteq (\overline{A} \cup B)$$
$$[(A_0 \wedge X_0) \rightarrow B_0 = (\overline{A_0} \vee \overline{X_0} \vee B_0) = X_0 \rightarrow (\overline{A_0} \vee B_0 ) ]$$

Вторая часть также преобразуется 

$$  B \subseteq A \cap X$$ 

Значит, $$ B \subseteq A , B \subseteq X$$

Из двух частей получаем ограничения на множество $X$.

$$B \subseteq X \subseteq (\overline{A} \cup B)$$ 

Аналогично разберем второе уравнение из системы

$$ A \cup X = C  \iff A \cup X \subseteq C , C \subseteq A \cup X$$

Первая часть преобразуется в 
$$A \cup X \subseteq C \iff A \subseteq C, X \subseteq C$$

Вторая часть преобразуется (используя обозначения)
$$C \subseteq A \cup X = (\overline{A} \cap C) \subseteq X $$
$$[C_0 \rightarrow (A_0 \vee X_0) = (\overline{C_0} \vee A_0 \vee X_0) = (\overline{A_0} \wedge C_0 ) \rightarrow X_0 ]$$

Из двух частей получаем ограничения на множество $X$.

$$C \cap \overline{A} \subseteq X \subseteq C$$ 

Получаем систему 

$$\begin{cases} B \subseteq X \subseteq (\overline{A} \cup B) \\ C \cap \overline{A} \subseteq X \subseteq C \end{cases}$$

Получаем ответ:

$$ B \cup (C \cap \overline{A}) \subseteq X \subseteq C \cap (\overline{A} \cup B)$$

Однако, учитывая условие $B \subseteq A \subseteq C$, оказывается, что левая и правая части равны, тогда получаем ответ

$$X = (C \setminus A) \cup B$$


\end{document}
