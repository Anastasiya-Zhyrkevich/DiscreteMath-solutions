\documentclass{article}
\usepackage[margin=0.25in]{geometry}
\usepackage[utf8]{inputenc}


\usepackage[T2A]{fontenc}
\usepackage[utf8]{inputenc}
\usepackage[russian]{babel}
\usepackage{amssymb}
\usepackage{multicol}
\usepackage{amsmath}
\usepackage{tikz}

\begin{document}

\textbf{ Принцип Дирихле }

$n$ -коробок для голубей, $n + 1$ голубь

Как ни сади голубей по коробкам, найдется коробка в которой как минимум два голубя.

\line(1, 0){500}

\textbf{13}

Докажите, что в любом множестве из 52 целых чисел найдутся по крайней мере два числа, сумма или разность которых делится на 100.

\textbf{Решение:}

Пусть множество из 52 чисел -- множество $X$. 

Рассмотрим множество, состоящее из остатков элементов множества  $X$ на 100. 

$$ A = \{ x \% 100 | x \in X \}$$

Если в множестве среди чисел в множестве $X$ было два числа, с одинаковыми остатками при делении на 100, то разность этих чисел делится на 100. 

Предположим, что таких нет.

Значит, в множестве $A$ 52 различных числа  из промежутка $[0, 99]$. 

Рассмотрим пары, которые в сумме дают 100 $$(1, 99), (2, 98), ... (50, 51)$$

По принципу Дирихле, если взять 51 число, то получится, что в хотя бы одной паре взяты оба числа. 

А тогда сумма соответствующих чисел из $X$ делится на 100.

\line(1, 0){500}

\textbf{14}

Точка $(x, y, z) \in R^3$ называется \textit{целой}, если $x, y, z \in Z$.

Докажите, что среди девяти целых точек найдутся по крайней мере две точки, для которых середина отрезка с концами в этой точке, также является целой точкой. 

\textbf{Решение:}

Рассмотрим, четность каждого компонента в данных 9 точках. 

Всего существует 8 различных комбинаций: 

\begin{itemize}
    \item (Ч, Ч, Ч)
    \item (Ч, Ч, Н)
    \item (Ч, Н, Ч)
    \item (Ч, Н, Н)
    \item (Н, Ч, Ч)
    \item (Н, Ч, Н)
    \item (Н, Н, Ч)
    \item (Н, Н, Н)
\end{itemize}

Значит, по признаку Дирихле, будет 2 точки с одинаковой расстановкой четности. 

Пусть это $(x_1, y_1, z_1)$, $(x_2, y_2, z_2)$. Тогда, все величины $x_1 + x_2$, $y_1 + y_2$, $z_1 + z_2$  являются четными числами, а тогда

$$\left(\frac{x_1 + x_2}{2}, \frac{y_1 + y_2}{2}, \frac{z_1 + z_2}{2}\right)$$

является \textit{целой} точкой.

\line(1, 0){500}

\textbf{15}

Докажите, что любое подмножество $S \subset \{1, 2, ... 200\}$ мошности $|S | = 101$ содержит по крайней мере два взаимно простых числа $x$, $y$, т.е. $HOD(x, y) = 1$ 

\textbf{Решение}

\textit{Лемма}
Два последовательных целых числа $x, x + 1$ являются взаимно простыми,  т.е. $HOD(x, x + 1) = 1$

\textit{Доказательство}

Пусть $HOD(x, x + 1) = d$. Тогда $x$  делится на $d$, $(x + 1)$ делится на $d$. Тогда и разность величин тоже делится на $d$, $(x + 1) - x = 1$ делится на $d$. 

Значит, $d = 1$.

\textit{Конец доказательства}

Разобьем все числа множества $\{1, 2, ... 200\}$  на $100$ пар последовательных чисел: 
$$ \{ (1, 2), (3, 4,), ... (199, 200) \}$$

По принципу Дирихле, если выбрать 101 элементов, то будет существовать 2 числа, которые из одной пары. 

Эти числа и будут взаимн простыми, необходимыми в условии задачи.

\line(1, 0){500}

\textbf{16}

Докажите, что любое подмножество $S \subset \{1, 2, ... 200\}$ мошности $|S | = 101$ содержит по крайней мере два числа $x$, $y$, что либо $x | y$, либо $y | x$.

\textbf{Решение}


\textit{Лемма}
Если у двух чисел совпадают наибольший нечетный делитель, то одно из чисел делится на другое. 

\textit{Доказательство}

Пусть, наибольший нечетный делитель $P$.

Первое число расписывается в виде 

$$x = 2^a \cdot P$$

Второе число расписывается в виде

$$y = 2 ^ b \cdot P$$

Очевидно, что меньшее число делится на большее.

\textit{Конец доказательства}

Разобьем все числа множества $$\{1, 2, ... 200\}$$  на группы относительно наибольшего нечетного делителя числа. 

$$\{ (1, 2), (3, 6), (5, 10), (7, 14), ...(99, 198),  (101) ... (199) \}$$

Всего групп -  $100$.

По принципу Дирихле, если выбрать 101 элементов, то будет существовать 2 числа, которые из одного множества. 
 
$$2^a \cdot P$$

где $P$ -- нечетное 

Пусть 2 числа, $a \le b$

$$2^a \cdot P$$

$$2^b \cdot P$$

$$HOD(2^a \cdot P, 2^b \cdot P) = 2^a \cdot P$$

$$2^a \cdot P  | 2^b \cdot P$$

Если $P$ одинаковый, то меньшее делит большее


Коробки 

\begin{itemize}
    \item $P = 1$ : $\{ 1, 2, 4, 8, 16, ... 128 \}$
    \item $P = 3$ : $\{ 3, 6, 12, 24, 48, 192 \}$
    \item $P = 5$  : $\{ 5, 10, 20, 40, 80, 160 \}$
    \item $P = 7$  : $\{ 7, 14, 28, ... \}$
    \item ...
    \item $P = 199$  : $\{199 \}$
\end{itemize}

Коробки не пересекаются по содержимому

Коробки 100

В одной коробке будет хотя бы 2 элемента. 
 
 
\line(1, 0){500}

\textbf{17}

Докажите, что для любого нечётного натурального числа  $m$ существует такое натуральное число $n$, что  $2^n - 1$  делится на $m$.

\textbf{Решение}

\textbf{Способ 1}

Рассмотрим числа  $$2^0 - 1,  2^1 - 1,  ...,  2^m - 1$$

Этих чисел  $m + 1$.  Какие-то два из них дают одинаковые остатки при делении на $m$, потому что различных таких остатков существует всего $m$. 

Пусть, скажем, числа  $2^k - 1$  и  $2^p - 1$  дают одинаковые остатки при делении на $m$ и  $k < p$.  

Тогда число  $$(2^p - 1) - (2^k -1) = 2^k(2^{p-k} - 1)$$  делится на $m$ и, поскольку $m$ нечётно,  $2^{p - k} - 1$  делится на $m$.

\textbf{Способ 2}

От противного. 

Пусть $2^a - 1$ никогда не делится на  $m$

Тогда, остатки $$ 1, ... m- 1$$

Существует $a, b$:

$$2^a - 1, 2^b - 1$$ 
дают одинаковые остатки на $m$.

Тогда разность делится на $m$
$$2^a - 1 - (2^b - 1) = 2^b (2^{a -b} - 1)$$

$2^b (2^{a -b} - 1)$ делится на $m$

По условию, $m$ -- нечетное

$HOD(2^ b, m) = 1$ 

Значит, $2^{a -b} - 1$ делится на $m$.

Получили противоречие.



\line(1, 0){500}



\textbf{18}

Каждый день на протяжении четырехнедельного отпуска отдыхающий

играл по крайней мере одну партию в шахматы. Общее число сыгранных партий не 

превышает $40$. Докажите, что надйтеся промежуток времени, 

состоящий из последовательный дней, в течении которых 

было сыграно ровно  $15 $ партий. 

\textbf{Решение}

$$A = [1, 4, [2, 5, 7, 1,] 3, 4  ... ]$$

Есть массив из 28 элементов. Нужно найти подмассив сумма элементов = 15.

Префикс сумм -- массив размера 29.

$$S_0 = 0$$
$$ S_{i + 1} = sum_{j = 0}^{j = i} A_j$$ 

$$ S = [0, 1, 5, 7, 12, 19, 20, .. ] $$ 

$$ S_0 = 0$$
$$S_1 = A_0 = 1$$
$$S_2 = A_0 + A_1 = 5$$
$$S_3 = A_0 + A_1 + A_2 = 7$$
$$S_4 = A_0 + A_1 + A_2 + A_3 = 12$$
$$S_5 = A_0 + A_1 + A_2 + A_3 + A_4 = 19$$
$$S_6 = A_0 + A_1 + A_2 + A_3 + A_4 + A_5= 20$$


$$ A_2 + A_3 + A_4  + A_5 = (A_0 + A_1 + A_2 + A_3 + A_4 + A_5) - (A_0 + A_1 ) = S_6 - S_2$$

Сумма на любом подотрезке вычисляется быстро !!


$$ A_i + A_{i + 1} + ... + A_{j} = S_{j + 1} - S_i$$

$$ S = [0, 1, 5, 7, 12, 19, 20, .. ] $$ 
Строите массив $S^`$ ОСТАТКИ ОТ ДЕЛЕНИЯ НА 15

$$ S = [0, 1, 5, 7, 12, 4, 5, .. ] $$ 

Получили две пятерки

$$ A_0 + A_1 $$ - делилась на 15 с остатком 5.

$$A_0 + A_1 + A_2 + A_3 + A_4 + A_5$$ - делилась на 15 с остатком 5.

Если рассмотреть разнность этих велчин, то получится последовательность элементов в массиве, сумма которых делится на 15.


\textbf{Вопрос на понимание}

Сложность вычисление 

Задача: Вам дан массив. Нужно выписать в ряд все суммы на подотрезках. 

$$ A = [1, 0, 7]$$

$$[1, 1, 8, 0, 7, 7]$$

$$[1]$$
$$[1 + 0]$$
$$[1 + 0 + 7]$$

$$[0]$$
$$[0 + 7]$$

$$[7]$$

За сколько Вы умеете строить ? - $n ^2$

$$A = [1,2, 3, 4, 5,6, 7, 8,9]$$

Многократное обращение.

$n$ - размер $A$
$m$ - запросов.
 
Решение :
Подсчитать один раз S

$func(i, j) = A_i + A_{i+ 1} + ... A_j = S_{j + 1} - S_i$

$n + m$ операций


А если бы честно считалли сумму каждый раз, то было бы 
$ m \cdot n$ операций


\end{document} 
