\documentclass{article}
\usepackage[margin=0.25in]{geometry}
\usepackage[utf8]{inputenc}


\usepackage[T2A]{fontenc}
\usepackage[utf8]{inputenc}
\usepackage[russian]{babel}
\usepackage{amssymb}
\usepackage{multicol}
\usepackage{amsmath}
\usepackage{tikz}

\begin{document}

\textbf{13}

Докажите, что в любом множестве из 52 целых чисел найдутся по крайней мере два числа, сумма или разность которых делится на 100.

\textbf{Решение:}

Пусть множество из 52 чисел -- множество $X$. 

Рассмотрим множество, состоящее из остатков элементов множества  $X$ на 100. 

$$ A = \{ x \% 100 | x \in X \}$$

Если в множестве среди чисел в множестве $X$ было два числа, с одинаковыми остатками при делении на 100, то разность этих чисел делится на 100. 

Предположим, что таких нет.

Значит, в множестве $A$ 52 различных числа  из промежутка $[0, 99]$. 

Рассмотрим пары, которые в сумме дают 100 $$(1, 99), (2, 98), ... (50, 51)$$

По принципу Дирихле, если взять 51 число, то получится, что в хотя бы одной паре взяты оба числа. 

А тогда сумма соответствующих чисел из $X$ делится на 100.

\newpage

Точка $(x, y, z) \in R^3$ называется \textit{целой}, если $x, y, z \in Z$.

Докажите, что среди девяти целых точек найдутся по крайней мере две точки, для которых середина отрезка с концами в этой точке, также является целой точкой. 

\textbf{Решение:}

Рассмотрим, четность каждого компонента в данных 9 точках. 

Всего существует 8 различных комбинаций: 

\begin{itemize}
    \item (Ч, Ч, Ч)
    \item (Ч, Ч, Н)
    \item (Ч, Н, Ч)
    \item (Ч, Н, Н)
    \item (Н, Ч, Ч)
    \item (Н, Ч, Н)
    \item (Н, Н, Ч)
    \item (Н, Н, Н)
\end{itemize}

Значит, по признаку Дирихле, будет 2 точки с одинаковой расстановкой четности. 

Пусть это $(x_1, y_1, z_1)$, $(x_2, y_2, z_2)$. Тогда, все величины $x_1 + x_2$, $y_1 + y_2$, $z_1 + z_2$  являются четными числами, а тогда

$$\left(\frac{x_1 + x_2}{2}, \frac{y_1 + y_2}{2}, \frac{z_1 + z_2}{2}\right)$$

является \textit{целой} точкой.
\end{document}
